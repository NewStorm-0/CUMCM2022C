\documentclass[withoutpreface,bwprint,draft]{cumcmthesis}

\begin{document}
\section{1}
\subsection{1}
\subsection{1}
\paragraph{1}
确定比较对象(评价对象)和参考数列(评价标准)。设评价对象有 $m$ 个,评价指标
有 $n$ 个,参考数列为 $x_0 = \{x_0(k)|k=1,2,\dots,n\}$,比较数列为 $x_i = 
\{x_i(k)|k=1,2,\dots,n\}, i=1,2,\dots,m$。

第二步,可以通过熵权法、层次分析法等确定权重,一般这里取等权重即可。

要想确定各指标值对应的权重,可用层次分析法等方法来确定各指标对应的权重 $w = 
[w_1,\dots,w_n]$,其中 $w_k(k = 1,2,\dots,n)$为第 $k$ 个评价指标对应的权重。

第三步,计算灰色关联系数。其中分辨系数一般取 0.5 即可

计算灰色关联系数:
\[
    \xi_i(k)=\frac{\displaystyle \min_s\min_t|x_0(t)-x_s(t)|+\rho\max_s
    \max_t|x_0(t)-x_s(t)|}{\displaystyle |x_0(k)-x_i(k)|+\rho\max_s
    \max_t|x_0(t)-x_s(t)|}
\]

为比较数列 $x_i$ 对参考数列 $x_0$ 在第 $k$ 个指标上的关联系数,其中 $p\in[0,1]$ 
为分辨系数。其中,称 $\min_s\min_t|x_0(t)-x_s(t)|$,$\max_s\max_t|x_0(t)-
x_s(t)|$ 分别为两级最小差及两级最大差。

一般来讲,分辨系数 $\rho$ 越大,分辨率越大; $\rho$ 越小,分辨率越小。

第四步,计算加权关联度。

计算灰色加权关联度。灰色加权关联度的计算公式为
\[
    r_i=\sum_{k=1}^nw_i\xi_i (k)
\]
式中 $r_i$ 为第 $i$ 个评价对象对理想对象的灰色加权关联度。

第五步,评价分析。

评价分析。根据灰色加权关联度的大小,对各评价对象进行排序,可建立评价
对象的关联序,关联度越大,其评价结果越好。

\end{document}